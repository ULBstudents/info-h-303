\section{janvier 2008}

\subsection{question 1}
	Une manière de supprimer la généralisation dans un modèle  entité-association (EA) est de 	conserver la super-entité, d'y ajouter les attributs et les relations des sous-entités, d'y ajouter un attribut 'type' et ajouter toutes les contraintes nécessaires.	Donner deux autres manières de supprimer la généralisation dans un modèle EA. Spécifier et justifier pour chaque type de généralisation si chacun des deux cas peut ou non s'appliquer.


\paragraph{(tentative de) réponse :}
Les deux autres façons sont : 
\begin{itemize}
	\item Conserver la sous-identité 
	\item Modéliser avec des relations ordinaires
\end{itemize}
(slides chapitre 5 du cours)
\paragraph{Conserver la sous-identité :} Propage à chaque sous-entité les attributs 
(et identificateurs) et les relations de la super-entité.\\
Il y a redondance des attributs et relations. Les applications deviennent plus 
complexes, et il y a des contraintes inter-relations.

Cette méthode n'est applicable que dans le cas de l'héritage total et exclusif.

\paragraph{Modéliser avec des relations ordinaires :}
On créé de nouvelles relations. Elles sont mandataires de la sous-entité et 
optionnelles pour la super-entité.
Les sous-identités deviennent des entités faibles, la super-entité devient 
une simple entité.
Les attributs et anciennes relations sont conservées.\\

On ajoute de la redondance (les nouvelles relations ont la m\^eme signification).
Les contraintes doivent exprimer le type de généralisation (t, e), 
certaines opérations deviennent plus complexes.


\subsection{question 2}
	Donner et expliquer les trois manières vues au cours de gérer la suppression en conservant l'intégrité référentielle dans le modèle relationnel. Illustrer à l'aide d'exemples bien choisis chacun des cas.



\subsection{question 3}
	Définir l'opération de division de l'algèbre relationnelle. Donner un équivalent en calcul relationnel tuple.
\subsection{question 4}
Donner un exemple bien choisi d’utilisation de la jointure externe (outer join).

\subsection{question 5}
Définir troisième forme normale et forme normale de Boyce-Codd.
Donner un exemple de relation qui est:
\begin{itemize}
	\item en deuxième mais pas en troisième forme normale,
	\item en troisième mais pas en forme normale de Boyce-Codd.
\end{itemize}
Donner les anomalies de mise à jour pour chacun des deux exemples.


\section{Janvier 2009}

\subsection{question 1}
A l’aide d’un exemple, expliquer comment remplacer une association ternaire par un modèle
équivalent sans relation ternaire. Donner la traduction relationnelle des 2 modélisations.

\subsection{question 2}
Expliquer la notion de contrainte d’intégrité du modèle relationnel. Citer et expliquer brièvement
les différents types de contraintes dans le modèle relationnel : celles présentes dans le modèle
et celles annexées à celui-ci.

\subsection{question 3}

Définir l’opération de produit cartésien en algèbre relationnelle. Expliquer le lien entre cette
opération et la jointure, la jointure naturelle et la jointure externe.

\paragraph{Réponse :} Slides chapitre 6, Relational Algebra\\
Le produit cartésien associe chaque tuple d'une relation avec 
tous les tuples d'une autre. \\
La jointure associe deux relations sur base d'une condition : \\
La jointure naturelle lie deux relations sur la condition d'égalité de deux 
attributs (equijoin), et ne retourne qu'un des deux champs :\\
$T*_{A=B}S$ ne va retourner que A. \\
$R*S$ va tester l'égalité des attributs qui ont le même nom.\\

La jointure externe, quant à elle, sélectionner les éléments d'une table 
indépendamment du fait qu'ils soient contenus ou non dans la deuxième table. 




\subsection{question 4}
Expliquer la séquence intuitive d’exécution (sémantique opérationnelle) d’une requête de type :
SELECT (...) FROM (...) WHERE (...) GROUP BY (...) HAVING (...)

\subsection{question 5}
Définir la notion de dépendance multi-valuée. Définir également la quatrième forme normale.
Expliquer le lien entre les dépendances multi-valuées et la première forme normale.


\section{Juin 2013}

\subsection{Modèle relationnel}
Une manière de supprimer la généralisation dans un modèle entité-association (EA)
est de conserver la super-entité, d'y ajouter les attributs et les relations 
des sous-entités, d'y ajouter un attribut "type", et ajouter toutes les 
contraintes nécessaires. Donner deux autres manières de supprimer la 
généralisation dans un modèle EA. Spécifier et justifier pour chaque type de généralisation 
si chacun des deux cas peut ou non s'appliquer. 

\paragraph{redo :}
Les deux autres façons sont : 
\begin{itemize}
	\item Conserver la sous-identité 
	\item Modéliser avec des relations ordinaires
\end{itemize}
(slides chapitre 5 du cours)
\paragraph{Conserver la sous-identité :} Propage à chaque sous-entité les attributs 
(et identificateurs) et les relations de la super-entité.\\
Il y a redondance des attributs et relations. Les applications deviennent plus 
complexes, et il y a des contraintes inter-relations.

Cette méthode n'est applicable que dans le cas de l'héritage total et exclusif.

\paragraph{Modéliser avec des relations ordinaires :}
On créé de nouvelles relations. Elles sont mandataires de la sous-entité et 
optionnelles pour la super-entité.
Les sous-identités deviennent des entités faibles, la super-entité devient 
une simple entité.
Les attributs et anciennes relations sont conservées.\\

On ajoute de la redondance (les nouvelles relations ont la m\^eme signification).
Les contraintes doivent exprimer le type de généralisation (t, e), 
certaines opérations deviennent plus complexes.

\subsection{Calcul tuple}

(a) Donner la forme générale d'une requête en calcul relationnel tuple.\\
(b) Le quantificateur universel ($\forall$) est souvent associé à l'implication 
($\implies$). Expliquer.

\subsection{SQL}
Donner la syntaxe et le contexte dans lequel la clause HAVING du langage SQL peut être 
utilisée. Illustrez par un exemple.

\subsection{Normalisation}
Définissez la notion de \textit{dépendance fonctionnelle}. \\
Définissez les trois premières formes normales et donnez trois exemples 
illustrant ces différentes formes.

\section{Juin 2015}

\subsection{Modèle entité association }

Donnez 3 manières de supprimer la généralisation dans un modèle entité-association.\\
Pour chacune de ces trois manières, spécifiez si elle est adaptée ou non lorsque la 
généralisation est du type \textit{partiel}.\\
Quelle contrainte supplémentaire doit-on ajouter lorsque la généralisation est 
de type \textit{exclusif} ?

%-----------------------------------------------------------------------------------
%----- Answer ----------------------------------------------------------------------
\paragraph{réponse incomplète}
Les trois façons sont : 
\begin{itemize}
	\item Conserver la super-entité
	\item Conserver la sous-identité 
	\item Modéliser avec des relations ordinaires
\end{itemize}
(slides chapitre 5 du cours)

\paragraph{Conserver la super-entité :}
On ajoute les attributs et les relations des sous-entités, 
on ajoute un attribut 'type' et 
on ajoute toutes les contraintes nécessaires.

\paragraph{Conserver la sous-identité :} Propage à chaque sous-entité les attributs 
(et identificateurs) et les relations de la super-entité.\\
Il y a redondance des attributs et relations. Les applications deviennent plus 
complexes, et il y a des contraintes inter-relations.

Cette méthode n'est applicable que dans le cas de l'héritage total et exclusif.

\paragraph{Modéliser avec des relations ordinaires :}
On créé de nouvelles relations. Elles sont mandataires de la sous-entité et 
optionnelles pour la super-entité.
Les sous-identités deviennent des entités faibles, la super-entité devient 
une simple entité.
Les attributs et anciennes relations sont conservées.\\

On ajoute de la redondance (les nouvelles relations ont la m\^eme signification).
Les contraintes doivent exprimer le type de généralisation (t, e), 
certaines opérations deviennent plus complexes.

\subsection{Algèbre relationnelle}
Expliquer comment se passer de l'opération de division en algèbre relationnelle.

\paragraph{slides chapitre 5, fin}
la division peut être remplacée par la différence et le produit cartésien :
$R \div S = T$ est équivalent à \\
$T1 \leftarrow \pi_{A}(R)$\\
$T2 \leftarrow \pi\_{A}((T1 x S) -R)$\\
$T \leftarrow T1 - T2 $\\

\subsection{SQL}
Soit les relations Employee(\underline{SSN}, Name, Salary) et WorksOn(\underline{ESSN, PNO}, Hours), 
où WorksOn.ESSN référence Employee.SSN.
Parmi les trois requêtes suivantes, lesquelles expriment la question : 
\textit{"Donner la moyenne des salaires des employés qui travaillent plus de 10 heures sur un projet ? "} Justifiez à l'aide d'un exemple les requêtes incorrectes.\\
\begin{enumerate}
\item 
\begin{lstlisting}
SELECT AVG(Salary) 
FROM Employee, WorksOn
WHERE SSN = ESSN AND Hours > 10 
\end{lstlisting}
\item 
\begin{lstlisting}
SELECT AVG(DISTINCT Salary) 
FROM Employee, WorksOn
WHERE SSN = ESSN AND Hours > 10 
\end{lstlisting}	
\item 
\begin{lstlisting}
SELECT AVG(DISTINCT Salary) 
FROM Employee, WorksOn
WHERE SSN IN 
	(SELECT ESSN 
	FROM WorksOn 
	WHERE Hours > 10) 
\end{lstlisting}
\end{enumerate}



\subsection{Normalisation}
Définissez la notion de \textit{Dépendance fonctionnelle}.\\
Définissez les trois premières formes normales et donnez trois exemples 
illustrant ces différentes formes.


