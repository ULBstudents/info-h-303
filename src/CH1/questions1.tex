\section{Questions chapitre 1}

\subsection{Définir les termes}

\begin{itemize}
	\item Donnée (data)
	\item Base de données (database)
	\item SGBD (DBMS)
	\item Système de bases de données (database system)
	\item Database catalog
	\item Program-data independance
	\item User View
	\item Administrateur de base de données (DBA)
	\item Utilisateur final (end user)
	\item Canned Transaction
	\item Persistent object
	\item meta-data
	\item transaction processing application
\end{itemize}

\noindent
Rep (prop) :\\ 
\textbf{Database} : une collection de données connexes; 
Une DB ne représente qu'une partie du monde réel. On peut parler de 
"miniworld" ou "Universe of Discourse" (UoD). \\
\textbf{Data} : des faits connus qui peuvent être enregistrés et ont une signification 
implicite.\\
\textbf{SGBD} : Collection de programmes qui permettent de créer et de maintenir 
une base de données. 
Définir une base de données signifie d'en spécifier les types de données, 
les structures, et les contraintes pour les données qui seront stockées. 
Construire une base de données est le processus où l'on enregistre les données 
dans un espace de stockage contrôlé par le SGBD. Manipuler une db implique 
certaines fonctions comme "interroger" (querying) la db pour retrouver 
certaines données spécifiques, en mettre à jour (update) pour refléter 
les changements du miniworld, la génération de rapports à partir des données.\\
\textbf{Db System} : On appelle "database system" l'ensemble database et DBMS. \\
\textbf{Database catalog} : Le catalogue d'une base de données est 
l'ensemble des définitions des objets "bases de données", comme les 
tables, les vues (tables virtuelles), les indexes, utilisateurs et groupes, contraintes, etc. Les informations stockées dans le catalogue sont 
les "meta-data".\\
\textbf{Program-data independance} Dans le traitement de fichiers traditionnel, 
la structure des data files est intégrée dans le programme d'accès. 
Donc un changement dans la structure nécessite de changer tous les programmes
qui accèdent à ces fichiers. 
Au contraire, avec un SGBD, les programmes d'accès ne nécessitent pas de changement, 
la plupart du temps. C'est ce qu'on appelle "Program-data independance". \\
Cela permet de changer le schéma à un niveau, sans avoir à changer au niveau supérieur. 
\begin{itemize}
\item[indépendance logique des données] : Un schéma externe est isolé des changements 
qui ne concernent pas le community schema.
\item[indépendance physique des données] : Les schémas sont isolés des changements de données
\end{itemize}

Illustration 
\begin{itemize}
\item un lien entre deux enregistrements peut être conceptuel ou logique (pointeur)
\item l'ajout d'un index à une table ne modifie pas la programmation. La 
seule modification est l'efficacité.
\item Ajout de nouveaux champs : seuls les programmes désirant y accéder devront être modifiés
\end{itemize}

\textbf{User View} Typiquement, une DB possède plusieurs utilisateurs, et chacun d'eux 
peut avoir besoin de "vues/perspectives" différentes.  
Une vue peut être un sous-ensemble de la db, ou contenir des 
données virtuelles, dérivées de la db stockée, mais pas directement 
stockées elles-mêmes. \\
\textbf{Administrateur de base de données} : Dans un environnement de DB, 
la ressource primaire est la base de données elles-même, et la seconde 
ressource est le SGBD, (et les softwares relatifs). 
L'administration de ces ressources est la responsabilité de 
l'administrateur BD : DBA. 
Il donne accès à la db, pour coordonner et gérer l'usage, acquérir des programmes 
et ressources au besoin. \\
\textbf{End User} : Les personnes dont le job nécessite un accès à la db, 
pour interroger, mettre à jour, générer des rapports. C'est en général 
pour eux que la db existe. Il y a plusieurs types de end-users :
\begin{itemize}
	\item Casual end user (décontracté): Il accède peu souvent, mais il a besoin de certaines infos à chaque fois.
	\item Naive : Il met à jour souvent la db, au travers de "canned transactions" 
	(requêtes sans danger, qui ont été vérifiées et ne risquent pas de causer de pb)
	\item Sophistiqué : ingénieurs, scientifiques, etc. qui implémentent leurs solutions 
	et ont des besoins plus complexes
	\item Stand-alone (pas trop compris)
\end{itemize}
\textbf{canned transaction}: transaction en "conserve", standard, qui a été vérifiée, qui peut mettre à jour des données sans que l'on craigne pour l'intégrité de la base, toujours le même type de transaction.\\
\textbf{Persistent object} : Comme en POO, c'est un objet qui persiste 
même après fermeture du programme. Opposé à transient, qui est "transitoire", n'existe plus à la fin de 
l'exécution.\\
\textbf{meta-data} : ce qui est contenu dans le catalogue;\\
\textbf{transaction processing application} :
\textit{En informatique, et particulièrement dans les bases de données, 
une transaction telle qu'une réservation, un achat ou un paiement est mise en 
œuvre via une suite d'opérations qui font passer la base de données d'un état A 
 — antérieur à la transaction — à un état B postérieur et des mécanismes  permettent d'obtenir que cette suite soit à la fois atomique, cohérente, isolée et durable.} (wikipedia)

\subsection{Citer trois types différents d'actions qu'impliquent les bases de données ? Discuter chacune d'elles}
\begin{itemize}
	\item Interroger (querying)
	\item Modifier, la structure ou les données
	\item Analyser (report)
\end{itemize} 
... to be continued


\subsection{Discuter les principales caractéristiques de l'approche "Bases de données" et des différences par rapport à une gestion de fichier traditionnelle}

Dans la gestion de fichiers traditionnelle, chaque utilisateur définit 
ses propres fichiers et leur structure. Ainsi, cela peut être très difficile 
de centraliser les données. Par exemple, si chaque professeur stocke lui-même 
les notes des élèves, et que l'on souhaite comparer les notes de deux classes, 
on va avoir du mal à récupérer l'ensemble des notes de deux professeurs différents, 
car les fichiers seront structurés de façon différente.
Dans l'approche DB au contraire, la structure est définie une seule fois, et on 
évite la redondance. L'accès y est géré par utilisateur.

\paragraph{Description}Par rapport à un système de fichier, la DB contient une description 
de la structure et des contraintes. (catalogue). Dans un système de fichier, la 
définition des données fait partie de l'application elle-même. 

\paragraph{Indépendance}L'indépendance des données est également une différence : Dans un système de fichier, 
un changement de structure devra être répercuté sur chaque application.
En SGBD, la majeure partie du temps, cela ne provoquera aucun changement 
dans les applications, puisque le catalogue est stocké avec la DB.

\paragraph{Multiples utilisateurs et vues} Une DB peut fournir plusieurs 
types d'accès aux données, voire à des données "virtuelles". 

\paragraph{Accès et modifications multiples} Une grande différence par rapport 
au système de fichier traditionnel est que la DB permet un accès 
multiple et simultané. Un SGBD doit avoir un système de contrôle de concurrence 
(concurrency control) afin d'assurer l'intégrité des données. 
Par exemple, plusieurs réservations en même temps : on doit s'assurer que 
deux utilisateurs n'achètent pas la même place au même moment, qu'une 
seule place est vendue. (online transaction processing).


\subsection{Quelles sont les responsabilités du DBA et du Database Designer ?}
Le DBA donne accès à la db, pour coordonner et gérer l'usage, acquérir des programmes et ressources au besoin. Les DB Designers sont responsables du 
choix des structures pour stocker les données. Il doit comprendre les 
besoins afin de fournir la structure adaptée. La plupart du temps, 
le designer est dans l'équipe du DBA. 
Il devra communiquer avec tous les types d'utilisateurs/groupes et 
fournir des vues adaptées. 


\subsection{Quels sont les différents types d'utilisateurs finaux de bases de données (end users)?}
\begin{itemize}
	\item Casual end user (décontracté): Il accède peu souvent, mais il a besoin de certaines infos à chaque fois.
	\item Naive : Il met à jour souvent la db, au travers de "canned transactions" 
	(requêtes sans danger, qui ont été vérifiées et ne risquent pas de causer de pb)
	\item Sophistiqué : ingénieurs, scientifiques, etc. qui implémentent leurs solutions 
	et ont des besoins plus complexes
	\item Stand-alone (pas trop compris)
\end{itemize}


\subsection{Discuter des différentes "fonctionnalités" qui devraient être fournies par un SGBD}
