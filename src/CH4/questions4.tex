\section{Questions chapitre 4}

\subsection{What is a subclass ? When is a subclass needed in data modeling ?}

\subsection{Define the following terms :}

\begin{itemize}
	\item superclass of a subclass
	\item superclass/subclass relationship
	\item IS-A-relationship
	\item specialization
	\item generalization
	\item category
	\item specific (local) attribute
	\item specific relationships
\end{itemize}

\subsection{Discuss the mechanism of attribute/relationship inheritance. Why is it useful ?}

\subsection{Discuss user-defined and predicate-defined subclasses, and identify the differences between the two}

\subsection{Discuss user-defined and attribute-defined specializations, and identify the difference between the two}

\subsection{Discuss the two main types of constraints on specializations and generalizations}

\subsection{What is the difference between a specialization hierarchy and a specialization lattice ?}

\subsection{What is the difference between specialization and generalization ? Why do we not display this difference in schema diagrams ?}

\subsection{How does a category differ from a regular shared subclass ? What is a category used for ? Illustrate your answer with examples}

\subsection{For each of the following UML terms, discuss the corresponding term in the EER model, if any :}
\begin{itemize}
	\item object
	\item class
	\item association
	\item aggregation
	\item generalization
	\item multiplicity
	\item attributes
	\item discriminator
	\item link
	\item link attribute
	\item reflexive association
	\item qualified association
\end{itemize}

\subsection{Discuss the main differences between the notation for EER schema diagram and UML class diagrams by comparing how common concepts are represented in each}


\subsection{Discuss the two notations for specifying constraints on n-ary relationships, and what each can be used for}

\subsection{List the various data abstraction concepts and the corresponding modeling concepts in the EER model}

\subsection{What aggregation feature is missing from the EER model ? How can the EER model be further enhanced to support it ?}

\subsection{What are the main similarities and differenes between conceptual database modeling techniques and knowledge representation techniques}
