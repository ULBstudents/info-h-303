\section{Questions chapitre 2}

\subsection{Définir les termes}
\begin{itemize}
	\item Data Model
	\item Database Schema
	\item Database state
	\item Internal schema
	\item Conceptual schema
	\item External schema
	\item Indépendance des données
	\item DDL
	\item DML
	\item SDL
	\item VDL
	\item query language
	\item Database utility
\end{itemize}


\noindent
Rep (prop) :\\ 
\textbf{Data Model} : Collection de concepts qui peuvent être 
utilisés pour décrire la structure d'une base de données. Ils peuvent avoir 
des niveaux d'abstraction différents : haut niveau ou conceptuel ressemblent 
plus à ce que l'on propose aux utilisateurs habituels, 
bas niveau ou data model physique sont plutôt pour les spécialistes, plus complexes.
Entre les deux, il y a "representational" (ou implementation data model), qui 
cache un certain nombre de détails mais qui peut être implémenté directement dans
un système informatique.\\
note : heu ??? \\
\textbf{Database Schema} : 
Un schéma décrit l'organisation/structure d'une db. On distingue les données 
de leur structure. Cette structure est spécifiée par le design de la db.\\
\textbf{Database state} : L'état - aussi appelé snapshot, est la situation 
des données à un instant t. \\
\textbf{Internal schema} : 
Dans une architecture "three-schema architecture" (3 schémas, donc), 
le niveau "interne" décrit le stockage interne de la db : accès physique aux données, chemins d'accès, ... \\
\textbf{Conceptual schema}
Décrit la structure pour toute la communauté d'utilisateurs : 
cache les détails d'accès physiques aux données, et décrit les entités, 
les types de données, les relations, les opérations permises et les contraintes.
Un high-level model peut être utilisé à ce niveau.\\
\textbf{External schema} :
(vue) inclus des schémas externes ou des vues d'utilisateurs. Chaque schéma externe décrit une part de la db auquel un groupe/utilisateur est intéressé et cache le reste de la db.\\
\textbf{Indépendance des données}
L'architecture 3-schémas (décrite au dessus) peut être utilisée pour expliquer l'indépendance des données. On peut citer deux différentes sortes d'indépendance :
\begin{itemize}
	\item L'indépendance logique des données : La capacité de changer le schéma conceptuel sans devoir changer le schéma externe.
	\item L'indépendance physique des données : la capacité à changer le schéma interne sans devoir changer le modèle conceptuel (externe). 
\end{itemize}

\textbf{DDL} :
Data Definition Language, utilisé par le DBA pour définir le schéma.
Dans les SGBD avec une séparation entre le niveau conceptuel et les données, 
DLL est utilisé pour le schéma uniquement. \\
\textbf{DML} : 
La manipulation de la base de données directement se fait à l'aide du 
Data Manipulation Language.\\
\textbf{SDL} : Storage Definition Language, utilisé pour le schéma interne, lorsque la séparation conceptuel/data est moins nette.\\
\textbf{VDL} : View Definition Language. ?? \\
\textbf{query language} : Un langage de requête est utilisé pour accéder aux données stockées dans la db. \\
\textbf{Database utility} : 
Beaucoup de SGBD proposent des utilitaires avec ce type de fonctions :
\begin{itemize}
	\item Loading : charge des fichiers existants dans la db. On précise le 
	format de départ du fichier et le format désiré. L'utilitaire 
	va automatiquement convertir. 
	 \item Backup : Créé une copie de la db, par "dumping". Utilisé dans des 
	 cas extrêmes. La plupart du temps, on procède à des backup incrementiels : gain de place.
	 \item réorganisation de fichiers : Afin d'améliorer la performance, on peut souhaiter réorganiser.
	 \item Performance monitoring : Propose des statistiques.
\end{itemize}

\subsection{Discuss the main categories of data models}

\subsection{What is the difference between a database schema and a database state ?}

\subsection{Describe the three-schema architecture. Why do we need mappings between schema levels ? How do different shema definition languages support this architecture ?}

\subsection{What is the difference between logical data independence and physical data independence ?}
Data independence refers to the possibility to change the schema at one level without having to change it at the next higher level (nor having to change programs that access it at that higher level). The two types are:
\begin{itemize}
	\item \textbf{Logical data independence:} An external schema (and programs that access it) is insulated from changes that do not concern it in in the community schema (and in the physical schema)
	\item \textbf{Physical data independence:} the community and external schemas are insulated from changes in the physical schema
\end{itemize}
Logical data independence therefore refers to the fact that data will only be modified if it is logical to do so, if changes do not concern it, it is left untouched.
On the other hand physical data independence therefore refers to the fact that logical schemas of the data are unaffected by what the data actually is. If a person changes their name it will only affect the physical data of his name, it will not change the fact that he HAS a name.

\subsection{What is the difference between procedural and non-procedural DMLs?}

\subsection{Discuss the different types of users-friendly interfaces and the types of users who typically use each}

\subsection{With what other computer system software does a DBMS interact ?}

\subsection{Discuss some types of database utilities and tools and their fonctions}
