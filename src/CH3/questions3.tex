\section{Questions chapitre 3}

\subsection{Discuss the role of a high-level data model in the database design process}

\subsection{List the various cases where use of null value would be appropriate}

\subsection{Define the following terms :}

\begin{itemize}
	\item entité
	\item attribut
	\item valeur d'attribut
	\item instance de relation
	\item attribut composite 
	\item attribut multivalué
	\item attribut dérivé
	\item attribut complexe
	\item key attribute
	\item value set (domain)
\end{itemize}

\noindent
Rep (prop) :\\ 

\textbf{Entité :} 
Un objet important, un concept dans le monde réel (ma voiture rouge...) qui peut être conceptuel (travail), ou physique (voiture, ..) \\
Une entité est souvent le terme utilisé pour décrire une instance, mais aussi pour \textit{entity type, entity set, entity class}.
En pratique, dans ce cours, "entité" est similaire à ce qu'on entend par "objet".

\textbf{attribut :} \\

\textbf{valeur d'attribut :} \\

\textbf{instance de relation :}\\

\textbf{attribut composite :} \\

\textbf{attribut multivalué :} un attribut qui peut avoir plusieurs instances. 
Exemple : un élève qui suit 0..n cours, qui possède 0..n numéros de téléphone...\\

\textbf{attribut dérivé :} se déduit d'autres attributs. ex. nombre d'élèves dans une classe, âge d'une personne, statut majeur ou mineur...\\
\textbf{attribut complexe :} \\
\textbf{key attribute :} \\
\textbf{value set (domain) :} détermine des contraintes sur les données. Exemple : un numéro de téléphone fixe belge est composé de 9 chiffres\\

\subsection{Expliquer la différence entre un attribut et un set de valeurs}

\subsection{Qu'est-ce qu'un type de relation ? Expliquer la différence entre une instance de relation, un type de relation}
Un type de relation est conceptuel. Une instance est une entrée de cette relation dans une table.

\subsection{What is a participation role ? When is it necessary to use role names in the description oh a relationship types ?}

\subsection{Décrire deux alternatives pour spécifier les contraintes structurelles d'un type de relation. Quels sont les avantages et désavantages des deux ?}

\subsection{Under what conditions can an attribute of a binary relationship type be migrated to become an attribute of one of the participating entity types ?}

%\subsection{When we think of relationships as attribute, what are the value sets of these attributes ? What class of data models is based on this concept ?}

\subsection{Quelle est la signification d'une relation récursive ? Donner des exemples}

\subsection{When is the concept of a weak entity used in data modeling ? Define the terms} :
\begin{itemize}
	\item owner entity type
	\item weak entity type
	\item identifying relationship type
	\item partial key
\end{itemize}

\subsection{Est-ce qu'une faible entité peut être de degré supérieur à 2 ? Donner des exemples}
prop :
Oui. On peut imaginer une table dont les entrées représentent l'ajout d'un certain label (dont le texte est stocké sur une autre table) 
concernant un certain établissement (dont l'id est répertorié dans une autre table) apposé par un certain membre 
(dont l'id est stocké ailleurs). Dans ce cas, 
trois clés étrangères sont utilisées, et l'entité reste faible.

