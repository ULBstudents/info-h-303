\section{Questions sur le modèle relationnel (chap. 7 du livre)}

\subsection{Define the following terms :}
\begin{itemize}
	\item domain
	\item attribute
	\item n-tuple
	\item relation schema
	\item relation-state
	\item degree of a relation
	\item relational database schema
	\item relational database state
\end{itemize}


\subsection{Why are tuples in a relation not ordered ?}

\subsection{Why are duplicate tuples not allowed in a relation ?}

\subsection{What is the difference between a key and a superkey ?}

\subsection{Why do we designate one of the candidate key of a relation to be a primary key ?}

\subsection{Discuss the characteristics of relations that make the different from ordinary tables and files}

\subsection{Discuss the various reasons that lead to the occurence of null values in relations}

\subsection{Discuss the entity integrity and referential integrity constraints. Why is each considered important ?}

\subsection{Define foreign key. What is this concept used for N How does it play a role in the join operation ?}

\subsection{List the operations of a relational algebra and the purpose on each}

\subsection{What is union compatibility ? Why do the UNION, INTERSECTION and DIFFERENCE operations require that relations on which they are applied be union compatible ?}

\subsection{Discuss some types of queries for which renaming of attributes is necessary in order to specify the query unambiguously}

\subsection{Discuss the various types of JOIN operations. Why is theta join required ?}

\subsection{What is yhe FUNCTION operation ? What is used for ?}

\subsection{How are the OUTER JOIN operations different from the (inner) JOIN operations ? How is the OUTER UNION operation different from UNION ?}