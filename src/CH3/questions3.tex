\section{Questions chapitre 3 (livre)}

\subsection{Lister les différents cas où l'usage d'une valeur null peut être appropriée}
\begin{itemize}
\item L'attribut est optionnel (exemple : on possède ou pas un téléphone, ce n'est pas obligatoire)
\item L'attribut null permet d'enlever une généralisation (attribut "type" + autres optionnels)
\item ? 
\end{itemize}

\subsection{Définir les termes suivants :}

\begin{itemize}
	\item entité
	\item attribut
	\item valeur d'attribut
	\item instance de relation
	\item attribut composé 
	\item attribut multivalué
	\item attribut dérivé
	\item attribut complexe
	\item value set (domain)
	\item domaine
	\item relation
	\item clé
	\item clé primaire
\end{itemize}

\noindent
Rep (prop) :\\ 

\textbf{Entité :} 
Un objet important, un concept dans le monde réel (ma voiture rouge...) qui peut être conceptuel (travail), ou physique (voiture, ..) \\
Une entité est souvent le terme utilisé pour décrire une instance, mais aussi pour \textit{entity type, entity set, entity class}.
En pratique, dans ce cours, "entité" est similaire à ce qu'on entend par "objet".


\textbf{attribut :} Une entité ou une relation a des valeurs pour chaque attribut.
Celles-ci appartiennent à un set de valeurs, ou à un domaine.\\

\textbf{valeur d'attribut :} appartient à un set de valeurs (0 ou 1, par exemple), ou à un domaine\\

\textbf{instance de relation :} \\

\textbf{attribut composé :} Un attribut peut être composé de plusieurs champs, exemple : une adresse (city, nro, etc.) Conceptuellement, il s'agit du même attribut. On peut le spécifier dans la table en conservant le même préfixe (addr\_nro, addr\_city etc.)
\\

\textbf{attribut multivalué :} un attribut qui peut avoir plusieurs instances. 
Exemple : un élève qui suit 0..n cours, qui possède 0..n numéros de téléphone...\\

\textbf{attribut dérivé :} se déduit d'autres attributs. ex. nombre d'élèves dans une classe, âge d'une personne, statut majeur ou mineur...\\
\textbf{attribut complexe :} \\
\textbf{value set (domain) :} détermine l'ensemble des valeurs possibles. 
Exemple : un numéro de téléphone fixe belge est composé de 9 chiffres\\

\textbf{domaine :} Il donne de précieuses indications sur la structure de l'information. En SQL, en général, cela se limite au type de donnée traditionnel 
dans les langages de programmation : integer, real, etc. 

\textbf{Relation :}

\textbf{clé :} Lorsqu'un ensemble de clés possèdent la propriété d'identification d'un tuple, une clé est l'unité la plus petite de cet ensemble.
En général, une relation a plusieurs clés (clés candidates). 
La définition des clés appartient au schéma.
\textbf{clé primaire :} Les SGBD demandent souvent qu'une relation possède une clé primaire. Donc si la relation a plusieurs clés, l'une d'elle est 
privilégiée. 

Ne pas confondre "clé" et "index". Si un attribut est une clé, il n'y a pas forcément d'index. 
La clé est une notion conceptuelle, l'index est un concept physique utilisé dans le 
but de l'optimisation.


\subsection{Expliquer la différence entre un attribut et un set de valeurs}
Un attribut est conceptuellement ce que représente une valeur dans une 
colonne, un set de valeurs est ce que peut prendre cet attribut comme valeurs.
Exemple : une note peut aller de 0 à 20 (set), et cela correspond à la note (attribut)

\subsection{Qu'est-ce qu'un type de relation ? Expliquer la différence entre une instance de relation, un type de relation}
Un type de relation est conceptuel. Une instance est une entrée de cette relation dans une table.

\subsection{What is a participation role ? When is it necessary to use role names in the description oh a relationship types ?}


\subsection{Décrire deux alternatives pour spécifier les contraintes structurelles d'un type de relation. Quels sont les avantages et désavantages des deux ?}
Les contraintes peuvent être classifiées : 
\begin{itemize}
	\item clés
	\item dépendances
	\item intégrité référentielle
	\item contraintes "ad hoc" (spécifiques du domaine)
\end{itemize}
Une contrainte est tout ce qu'on veut exprimer au niveau de la 
structure de la db qui n'est pas possible d'exprimer 
à l'aide des mécanismes de description du data-model.

On peut spécifier une contrainte structurelle par une contrainte référentielle : 
une clé étrangère. Dans ce cas, le lien est symétrique. C'est une contrainte 
très présente dans les db. Cela signifie que la clé référencée 
doit exister. 

Des contraintes complexes peuvent exister : les salaires des employés doivent 
être inférieurs au salaire du supérieur. 
Dans ce cas, cela peut être spécifié dans l'application elle-même. 
Le problème est que le développeur doit connaître les contraintes, 
et que si elles changent, il faut modifier l'application.

Notons qu'il n'y a pas de différence fondamentale  de nature entre la structure de données et des contraintes. Cela dépend du sens que l'on souhaite donner.
Le même morceau d'information peut être considéré comme un fait ou une contrainte 
(les joueurs de telle équipe de foot habitent dans telle ville).

??
\subsection{Under what conditions can an attribute of a binary relationship type be migrated to become an attribute of one of the participating entity types ?}

%\subsection{When we think of relationships as attribute, what are the value sets of these attributes ? What class of data models is based on this concept ?}

\subsection{Quelle est la signification d'une relation récursive ? Donner des exemples}
C'est une relation qui porte sur le même type d'entité une fois ou plus.
Exemple : la supervision des employés. \\
Un employé peut superviser un employé, qui peut superviser un employé... (attention à la gestion des cycles)\\

\subsection{Quand utilise-t-on le concept de faible entité ? Définir les termes} :
\begin{itemize}
	\item owner entity type
	\item weak entity type
	\item identifying relationship type
	\item partial key
\end{itemize}

\subsection{Est-ce qu'une faible entité peut être de degré supérieur à 2 ? Donner des exemples}
prop :
Oui. On peut imaginer une table dont les entrées représentent l'ajout d'un certain label (dont le texte est stocké sur une autre table) 
concernant un certain établissement (dont l'id est répertorié dans une autre table) apposé par un certain membre 
(dont l'id est stocké ailleurs). Dans ce cas, 
trois clés étrangères sont utilisées, et l'entité reste faible.

